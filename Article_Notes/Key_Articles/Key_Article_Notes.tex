% This document holds a collection of the notes for the key articles in the flash drought identification
% and forecast thesis project.
%
% References within the notes are citations within the article the note is taken of.
% e.g., Otkin et al. (2014) in an itemize for Otkin et al. (2018) means to see the Otkin et al. (2014)
% reference within the Otkin et al. (2018) paper for further reading.
%
% The key articles are:
%   Christian et al. 2019 - On flash drought identification
%   Otkin et al. 2018 - On flash droughts in general

\documentclass[12pt, letterpaper]{article}

\usepackage{syntonly}
\usepackage{amsmath}

\syntaxonly % Use to ensure everything compiles. Speeds up compilation but suppresses output

\begin{document}
	Otkin et al. $2018$
	\begin{itemize}
		\item[-] NCEI for source on drought costs.
		\item[-] Examples in Introduction have a $3$ - $5$ category increase in $2$ months
		         (United States Drought Monitor (USDM) category).
		\item[-] Flash drought requires more than a precipitation deficit. Main features is evaporative 
		         stress to quickly deplete soils via ET.
		\begin{itemize}
			\item[-] Can be done with high temperatures and winds, sunny skies, and low relative humidity.
		\end{itemize}
	    \item[-] The term ``flash drought'' was coined by Svoboda et al. 2002 in the introduction to
	             the USDM.
	    \item[-] Otkin et al. $(2014, 2015a)$ for development of the  rapid change index (RCI),
	             which encases the accumulated magnitude of moisture stress changes over multiple weeks.
	    \begin{itemize}
	    	\item[-] RCI $< 0$ indicates drought is likely. The probability increases as the RCI
	    	         becomes more negative. 
	    \end{itemize}
        \item[-] Also consider the soil moisture index (SMI)
        \item[-] Soil moisture deficits starts at the surface and works down.
        \begin{itemize}
        	\item[-] This can be used to check models.
        \end{itemize}
        \item[-] Precipitation ($P$) $-$ potential evapotranspiration ($PET$) is a better predictor for
                 flash drought than precipitation and temperature.
        \begin{itemize}
        	\item[-] $P - PET$ indicates the balance between supply and demand for surface moisture.
        \end{itemize}
        \item[-] Rapidly declining soil moisture could be a precursor for flash drought.
        \item[-] See page 4, paragraph 3.
        \begin{itemize}
        	\item[-] Description on heat wave flash drought and precipitation flash drought from
        	         Mo and Lettenmaier ($2015$ and $2016$).
        \end{itemize}
        \item[-] For a flash drought definition, the rate of intensification is emphasized, as well
                 as the dry conditions.
        \par
        \item[-] First requirement: Drought indices used to monitor it should be computed over short
                 time scales ($< 1$ month) to monitor rapid changes and be sensitive to soil moisture,
                 ET, evaporative demand, or vegetation health.
        \item[-] Second requirement: Indices must fall into drought during rapid intensification (below
                 the $20^{\text{th}}$ percentile).
        \par
        \item[-] Suggests a suite of different magnitude and temperature change thresholds to classify
                 flash drought.
        \begin{itemize}
        	\item[-] E.g., change in $2$ USDM categories over $5$ weeks could be moderate flash drought.
        	\item[-] These should capture change in time and actually be sufficiently dry conditions.
        \end{itemize}
        \item[-] Flash droughts are more likely to occur when evaporative demand is above normal for
                 several weeks.
        \item[-] Consider the evaporative demand drought index (EDDI; Hobbins et al. $2016$; 
                 McEvoy et al. $2016$).
        \item[-] Standardized precipitation evapotranspiration index (SPEI) should be used.
        \item[-] Evaporative stress ratio $\left( ET/PET \right)$ can provide a clearer signal than PET.
        \item[-] EDDI and ESI compliment each other.
        \begin{itemize}
        	\item[-] EDDI identifies drought earlier, but has a higher false alarm rate.
        \end{itemize}
        \item[-] Flash drought can also be monitored with vegetation indices, NDVI, EVI, LSWI, etc. 
                 (i.e., monitoring the health of the vegetation).
        \item[-] New forecast methods that leverage the long-term memory of soil moisture should be
                 explored (see Lorenz et al. $2017$a, b).
        \item[-] Land-atmosphere interactions and their role in flash droughts should also be further
                 studied.
        \item[-] Easy to use and deliver drought forecasting tools are what are desired by effected 
                 groups (e.g., stakeholders).
	\end{itemize}
    \par
    Christian et al. $2019$
    \begin{itemize}
    	\item[-]
    \end{itemize}
\end{document}