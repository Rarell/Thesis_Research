% This document holds a collection of the notes for the articles in the flash drought identification
% and forecast thesis project.
%
% References within the notes are citations within the article the note is taken of.
% e.g., Otkin et al. (2014) in an itemize for Otkin et al. (2018) means to see the Otkin et al. (2014)
% reference within the Otkin et al. (2018) paper for further reading.
%
% The articles are:
%   Otkin et al. 2013 -- ESI Calculated from Thermal Infrared Satellite
%   Otkin et al. 2015 -- Early Drought Warning
%   Otkin et al. 2019 -- Vegetation and Soil Moisture Recovery
%   Basara et al. 2019 -- 2012 Drought
%   Wakefield et al. 2019 -- Land - Atmopshere
%   Mesinger et al. 2006 -- NARR
%   Gehne et al. 2019 -- GEFS Land Surface and State
%   Zsoter et al. 2019 -- NWP and Hydrology
%   Ek et al. 2003 -- Noah LSM
%   Chen et al. 1996 -- Transpiration Methods
%   Wang et al. 2018 -- Sensitivity of Ensemble Models to Bias
%   Zhou et al. 2017 -- GEFS Configuration

\documentclass[12pt, letterpaper]{article}

\usepackage{syntonly}
\usepackage{amsmath}

%\syntaxonly % Use to ensure everything compiles. Speeds up compilation but suppresses output

\begin{document}
	\underline{Otkin et al. $2013$ -- ESI Calculated from Thermal Infrared Satellite}
	\begin{itemize}
		\item[-] Palmer drought severity index (PDSI) responds to changes slowly.
		\begin{itemize}
			\item[-] It is better for long term drought.
			\item[-] Potentially oversensitive to temperature effects.
		\end{itemize}
	    \item[-] Theory for IR measurements:
	    \begin{itemize}
	    	\item[-] As moisture depletes, less energy is used for evaporation and transpiration.
	    	         As a consequence, the stressed canopy temperatures increase (compared to the
	    	         unstressed canopy).
	    \end{itemize}
        \item[-] IR measurements use thermal IR (TIR) based Atmosphere Land Exchange Inverse (ALEXI)
                 surface energy balance model estimates for ET.
        \begin{itemize}
        	\item[-] More details on page 3.
        \end{itemize}
        \item[-] Pages 3 -- 4 for details on ESI calculations.
        \item[-] ESI tested using 4 flash drought case studies with different vegetation, drought
                 intensities, and durations.
        \item[-] ESI is negative as drought increases.
        \item[-] All flash drought events were associated with positive temperature anomalies, and
                 low cloud cover.
        \begin{itemize}
        	\item[-] For the more severe cases, there was also persistent high winds, and high dewpoint
        	         depressions.
        \end{itemize}
        \item[-] Shortest composite and time differencing interval provided the earliest warnings as
                 they could respond more quickly to changing conditions.
        \item[-] Reductions in ET precede precipitation deficits and reductions in biomass.
        \begin{itemize}
        	\item[-] Hence, the ESI can provide an early warning system.
        \end{itemize}
        \item[-] ESI does not rely on precipitation observations and can be applied to data sparse
                 regions (due to being satellite based).
	\end{itemize}
    
    \underline{Otkin et al. 2015 -- Early Drought Warning}
    \begin{itemize}
    	\item[-] See the National Integrated Drought Information System (NIDIS) for defined goal in 
    	         creating a drought early warning system.
    	\begin{itemize}
    		\item[-] Provides a probabilistic forecast with sufficient spatial and temporal resolutions
    		         for users to make informed decisions.
    		\item[-] What resolution is sufficient?
    	\end{itemize}
        \item[-] Early warning systems should be integrated with new datasets that provide information on
                 subseasonal time scales ($< 3$ months) and updated weekly.
        \item[-] Section 3 on pages 3 -- 4 has an interesting discussion.
        \begin{itemize}
        	\item[-] Farming practices and how they impact drought and vice versa.
        \end{itemize}
        \item[-] ESI and RCI rated favorably in the group tests.
        \item[-] ESI and RCI were most helpful when used together.
        \begin{itemize}
        	\item[-] ESI evaluates current conditions.
        	\item[-] RCI provides a longer view on how conditions have been changing recently.
        \end{itemize}
        \item[-] There are some concerns over false alarms, but rated favorably for forecasts as
                 farmers and ranchers are accustomed to uncertainties in decision making.
        \begin{itemize}
        	\item[-] ``Any tool that can provide reliable information is potentially useful.''
        \end{itemize}
        \item[-] Change in ESI had more neutral feedback.
        \begin{itemize}
        	\item[-] Difficulty in interpreting it.
        	\item[-] Interesting discussion on pages 3 -- 4 (past paragraph on page 3).
        	\begin{itemize}
        		\item Change in ESI is normalized for usefulness, but the RCI was preferred for its
        		      easily understood format.
        		\item An explanation in on how to read a change in ESI would be useful.
        	\end{itemize}
        \end{itemize}
        \item[-] Plume diagrams had more mixed results.
        \begin{itemize}
        	\item[-] Improved in the second meeting as more time was taken to explain it and how to read
        	         it.
        	\item[-] Unfamiliar format and a lot of information is held within it.
        	\item[-] Difficult for non-scientists to read it unless clearly explained by via detailed
        	         examples.
        \end{itemize}
        \item[-] Several weeks advanced notice of drought is ideal for ranchers, and one or two months
                 for farmers.
        \item[-] See page 5 for benefits of an early warning drought system.
    \end{itemize}

    \underline{Otkin et al. 2019 -- Vegetation and Soil Moisture Recovery}
    \begin{itemize}
    	\item[-] Anderson et al. (2007$a$, $b$, 2013) for a detailed description of ESI.
    	\item[-] Myneni et al. (2002) for more information on MODIS LAI.
    	\item[-] Largely a case study over flash drought and recovery in 2015.
    	\item[-] Coined term ``flash recovery.''
    	\begin{itemize}
    		\item[-] Rapid improvement of conditions (likely due to heavy rainfall).
    	\end{itemize}
        \item[-] Also includes soil moisture estimates from land surface models and compares them.
        \item[-] In general, the ensemble average of the LSMs performed better for flash drought 
                 evolution than the individual models. 
        \item[-] Future work includes assessing LSMs during flash droughts.
        \item[-] Flash recovery may not translate to improved foliage and grain (plants may be too
                 damaged or go dormant).
    \end{itemize}

    \underline{Basara et al. 2019 -- 2012 Drought}
    \begin{itemize}
    	\item[-] Mesinger et al. 2006 for ET calculations in NARR.
    	\item[-] Additional use of other indices (HI and CTP). See Wakefield et al. 2019 for details.
    	\item[-] Overall, this is a study of the 2012 drought.
    	\item[-] Large flash drought event (set several records)
    	\begin{itemize}
    		\item[-] Might prove a useful example to test some things against, especially for some indices.
    	\end{itemize}
        \item[-] Drought is not a static, but evolves and propagates in time.
        \begin{itemize}
        	\item[-] This needs to be captured for any successful modeling.
        \end{itemize}
        \item[-] Note some of the impacts of drought -- modified air masses in new regions, helping the
                 drought to propagate.
    \end{itemize}

    \underline{Wakefield et al. 2019 -- Land - Atmopshere}
    \begin{itemize}
    	\item[-] Introduction has a good list of impacts of land - atmosphere coupling.
    	\item[-] Coupling strength is greatest over transition regions from humid to arid (several
    	         citations here).
    	\begin{itemize}
    		\item[-] Needs to be captured in modeling attempts.
    	\end{itemize}
        \item[-] Transition climates are more sensitive to ET changes in soil moisture and atmospheric
                 demand (more citations here).
        \item[-] The sign of the changes can depend on the temporal and spatial resolution of what is
                 used to examine them.
        \item[-] FWI of 0.7 is optimal for plant use, and 0.4 is water stress (Illston et al. 2008).
        \item[-] MetPy package in Python? (See May et al. 2017)
        \item[-] Two steps to determining CTP (convective triggering potential).
        \begin{itemize}
        	\item[$1)$] Locating the moist adiabat that intersects the temporal profile 100 mb above 
        	            the surface.
        	\item[$2)$] Integrating the area between the moist adiabat and temperature from 100 mb to
        	            300 mb above the surface.
            \begin{itemize}
        	    \item Essentially, a different type of CAPE.
            \end{itemize}
        \end{itemize}
        \item[-] HI (low level humidity index) is the sum of dewpoint depressions at 150 mb and 50 mb
                 above the surface.
        \begin{itemize}
        	\item[-] Note, only two levels are used for this ``average.'' Take with a grain of salt. 
        \end{itemize}
        \item[-] Interesting point made regarding NARR data (includes model data as well).
        \begin{itemize}
        	\item[-] NARR (and models) assimilates observations (such as radiosondes) into their data.
        	         So data located near observation sites may match observations more readily, while
        	         data sparse locations may be less realistic.
        \end{itemize}
        \item[-] CTP and HI have annual cycles, with summer having smaller variation.
        \begin{itemize}
        	\item[-] Result is that time scales less than 1 month have to be used.
        	\item[-] Changes in variability necessitate the need for daily standardized anomalies.
        \end{itemize}
        \item[-] Daily z-scores have small sample size (16 years, so only 16 samples).
        \begin{itemize}
        	\item[-] Resulting z-scores were non-Gaussian and, as a result, may be biased.
        	\item[-] z-scores were transformed into a Gaussian distribution for comparisons (similar to
        	         SPI). See McKee et al. 1993 and Wilks 2013 for details on this.
        \end{itemize}
        \item[-] Note: HI above normal (z-score $> 0$) is drier conditions.
        \item[-] Negative CTP in the climatology imply normally stable conditions.
        \begin{itemize}
        	\item[-] Not favorable for land - atmosphere coupling.
        	\item[-] This is why the paper focuses on the warm season (MJJAS) as OK has a positive CTP
        	         on average.
        \end{itemize}
        \item[-] Figure 8. 2015 does not have as clear of a signal (or dipole at least) in the quadrant
                 analysis as 2007 or drought years.
        \begin{itemize}
        	\item[-] See discussion on pages 13 -- 14.
        \end{itemize}
        \item[-] The percentage of days in quadrant 2 (low moisture and high instability/energy/heat
                 plumes) is greatest where the droughts are more intense (Figure 11).
        \item[-] CTP diagnoses instability, and HI the low level moisture.
        \item[-] Sensitivity of forecast skill to initial soil moisture conditions increases as the
                 magnitude of the soil moisture anomalies increases (Koster et al. 2011).
        \item[-] ET becomes more responsive as soils grow drier (Phillips and Klein 2014; 
                 Williams et al. 2016).
        \item[-] Found greater covariability between the surface and atmosphere during drought.
        \item[-] CTP and HI framework has some utility in diagnosing and predicting hydrologic extremes,
                 particularly for drought.
    \end{itemize}

    \underline{Mesinger et al. 2006 -- NARR}
    \begin{itemize}
    	\item[-] NARR covers a 25 year time span (1979 -- 2003, may be more by now).
    	\item[-] Uses a recent version of the Noah LSM.
    	\item[-] Claims to successfully assimilate high-quality and detailed precipitation observations.
    	\item[-] Also claims improved forcing to the LSM.
    	\item[-] Additionally, NARR provides an improved land hydrology and land - atmosphere interaction
    	         analysis.
    	\item[-] One of its goals was to better capture the regional hydrologic cycle (as well as other
    	         things) than the global reanalysis.
    	\item[-] NARR should also have a good representation of hydrologic extremes.
    	\item[-] NARR should also interface well with other hydrologic models.
    	\item[-] Several mixed fields are used as inputs to the LSM:
    	\begin{itemize}
    		\item[-] Land mask (land or water)
    		\item[-] Vegetation type
    		\item[-] Soil type
    		\item[-] Surface slop category
    		\item[-] Maximum snow albedo
    		\item[-] Soil column bottom temperature
    		\item[-] Number of root zone soil layers
    	\end{itemize}
        \item[-] Potentially the greatest addition to NARR was the assimilation of precipitation 
                 observations (from rain gauges).
        \item[-] 10m surface winds and 2m surface moisture observations were also assimilated.
        \item[-] 2m temperature observations, however, were excluded from NARR.
        \begin{itemize}
        	\item[-] They created issues with vertical profiles generated.
        \end{itemize}
        \item[-] For a practical point, most NARR users will use the analyzed variables, but will also
                 use the first guess for non-analyzed fields such as surface fluxes.
        \item[-] Figure 10.
        \begin{itemize}
        	\item[-] The residual of the moisture budget (which should be 0, i.e. a closed balance) is
        	         good in the central U.S. There is an imbalance, however, in the Intermountain West,
        	         and Coastal Plains.
        \end{itemize}
        \item[-] The Noah LSM simulates soil temperature and moisture (frozen included) in four soil
                 layers (10-, 30-, 60-, and 100-cm thicknesses).
        \begin{itemize}
        	\item[-] The infiltration scheme accounts for subgrid variability in soil moisture and
        	         precipitation.
        	\item[-] Surface evaporation includes evaporation from soils, transpiration from vegetation
        	         canopy, evaporation from dew/frost or canopy intercept precipitation, and snow
        	         sublimation.
        \end{itemize}
        \item[-] Land surface subset of NARR is available.
        \begin{itemize}
        	\item[-] Subset includes: surface fluxes and states, soil column states, and surface variables.
        \end{itemize}
        \item[-] Improvements in the NARR are generally greater in the winter than in the summer.
        \item[-] Overall, a good summary of the NARR, including its capabilities and limitations. 
                 Recommended for any use of the NARR.
    \end{itemize}

    \underline{Gehne et al. 2019 -- GEFS Land Surface and State}
    \begin{itemize}
    	\item[-] Some of the problems with the GEFS is that often the spread is too small, or the 
    	         uncertainty is too high.
    	\item[-] Hypothesis is that a major source of errors are a lack of treatment of uncertainties
    	         in the soil state and associated LSMs.
    	\item[-] Also hypothesizes that the spread can be increased by introducing stochastic surface
    	         parameters and states.
    	\item[-] \underline{The GEFS uses the Noah LSM}. (Ek et al. 2003 for details)
    	\item[-] The ensemble of the GFS accounts for the random error, but not the systematic error.
    	\begin{itemize}
    		\item[-] An ensemble should be bias corrected before its skill is evaluated (Wang et al. 2018).
    	\end{itemize}
        \item[-] Previous works show perturbations in soil moisture initial conditions increase 
                 precipitation spread in forests.
        \item[-] Perturbations in this study are done through EOFs of the normalized soil moisture state
                 estimates between two LSMs.
        \begin{itemize}
        	\item[-] Noah LSM
        	\item[-] Community Land Model (CLM) LSM
        \end{itemize}
        \item[-] Subgrid heterogeneity is not accounted for in the Noah LSM in the GEFS.
        \item[-] The Noah LSM also does not allow for multiple vegetation types within its grid.
                 Rather, it uses the dominant vegetation as its vegetation type.
        \item[-] Zhou et al. 2017 for GEFS configuration.
        \item[-] Section 2 for a discussion on the idea and physics behind several perturbation schemes.
        \item[-] Different LSMs use different parameters (e.g., soil conductivity), have different 
                 textures, are driven by different specific evaporation and runoff formulations.
        \begin{itemize}
        	\item[-] As a result, soil moisture is model specific in its mean and variability.
        	\item[-] To compare soil moisture between different LSMs, the soil moisture should be
        	         standardized/normalized.
        \end{itemize}
        \item[-] See Bonan et al. 2002 for the CLM LSM.
        \item[-] See pages 6 -- 8 for a discussion on model uncertainties and perturbations.
        \item[-] See pages 8 -- 9 for a verification method.
        \item[-] Forecast mean error is the root-mean-square-error (RMSE) between the ensemble mean and
                 the verification.
        \item[-] Equation 12 is an equation for the spread (standard deviation) of the ensemble.
        \item[-] Equation 13 for the RMSE equation.
        \item[-] Equations 14 -- 15 for the bias correction.
        \item[-] Impacts of the soil moisture perturbation are largest in the summer hemisphere.
        \item[-] Found a small increase in the spread.
        \item[-] Previous studies (Koster et al. 2006; Zhang et al. 2011) showed that soil perturbations
                 do not effect the atmosphere in the GFS as much as other models.
        \begin{itemize}
        	\item[-] Soil and atmosphere may be too weakly coupled.
        \end{itemize}
        \item[-] In the end, temperature bias was found to be a larger contributor to spread.
        \begin{itemize}
        	\item[-] Hence, addressing model bias is important for modeling efforts.
        \end{itemize}
    \end{itemize}

    \underline{Zsoter et al. 2019 -- NWP and Hydrology}
    \begin{itemize}
    	\item[-] LSMs provide vertical fluxes of energy and water between the land and atmosphere,
    	         but are generally less considered with other features (namely runoff).
    	\begin{itemize}
    		\item[-] Better forecasts, but less known for the hydrologic cycle.
    	\end{itemize}
        \item[-] The Euro uses a LSM and LDAS (land data assimilation system).
        \begin{itemize}
        	\item[-] May be worth looking into this for the GFS as well.
        \end{itemize}
        \item[-] Main issue is focused on an open water budget and associated issues.
        \begin{itemize}
        	\item[-] LDAS adds and removes water in its increments, which can leave an open water budget.
        \end{itemize}
        \item[-] Figure 5 is very interesting.
        \begin{itemize}
        	\item[-] Red indicates where the LDAS removes water, blue where the LDAS adds water.
        	\item[-] $5a$ for the whole hydrologic cycle, $5b$ for snow water equivalent, $5c$ for soil
        	         moisture.
        	\item[-] Section $3b$ (pages 8 -- 10) for reasoning why water is added/removed where it is.
        	\item[-] Note that while this figure is very interesting, it is for the Euro, not the GFS.
        \end{itemize}
        \item[-] Soil moisture can compensate for precipitation and 2m temperature biases.
        \begin{itemize}
        	\item[-] E.g., if there is a negative temperature bias (temperature is too low), the soil
        	         moisture assimilation will remove water, reducing evaporative cooling and increasing
        	         temperature.
            \item[-] Think back to Mesinger et al. 2006, using the first guess for non-analyzed fields.
            \item[-] I.e., order of operations matter. If bias correction is done after analysis (and
                     soil moisture compensated for this), then there is over compensation and an error
                     is introduced in soil moisture.
            \begin{itemize}
            	\item Doing a bias correction before the analysis might be more useful, but more 
            	      investigation is needed.
            \end{itemize}
            \item[-] Also worth seeing if this is a feature in other LSMs.
        \end{itemize}
        \item[-] An interesting problem occurs with snowmelt. It seems to occur too late, and the LDAS
                 method of handling it seems to result in no spring peak discharge from snowmelt.
        \begin{itemize}
        	\item[-] Worth seeing if this happens in other models or LSMs, or if it is a Euro, HTESSEL
        	         feature.
        	\item[-] See pages 10 -- 12 (section $3c$) for a discussion on it, and possible reasons.
        \end{itemize}
        \item[-] Figure 8.
        \begin{itemize}
        	\item[-] Note the differences in LDAS and non-LDAS are largest in the snow and river 
        	         discharge.
        	\item[-] Both tests seem to have a problem following the observed discharge, however.
        \end{itemize}
        \item[-] LDAS seems to decrease water in rivers. (Section $3e$, pages 14 -- 16, gives the
                 exact nature of estimates. E.g., where it over and underestimates, etc. Also see
                 Figure 9. They are omitted here as this study is for the Euro and possibly may not
                 apply as effectively to the Noah LSM.)
        \item[-] Large differences between LDAS and non-LDAS simulations were in snow-pack regions,
                 but in general the results were mixed with largely similar performances.
        \begin{itemize}
        	\item[-] Negative bias in snow cover.
        \end{itemize}
        \item[-] De Lannoy et al. (2012)
        \begin{itemize}
        	\item[-] Showed on a small catchment in the Colorado River (western U.S.), the season
        	         averaged snow pack is largely decreased by the snow water assimilation in the Noah
        	         LSM.
        \end{itemize}
        \item[-] See Arsenault et al. (2013) for a similar thing with the CLM LSM.
        \item[-] Errors in snow cover are potentially the largest error source for LSMs in the hydrologic
                 cycle.
        \item[-] Section $5b$ has an interesting discussion on hydrologic forecasting.
    \end{itemize}

    \underline{Ek et al. 2003 -- Noah LSM}
    \begin{itemize}
    	\item[-] The Noah LSM is, indeed, implemented in the GFS.
    	\item[-] Table 1 for a list of Noah LSM features and additions, and associated citations for
    	         technical and mathematical (hopefully) details.
    	\item[-] Chen et al. (1996, section 3.1.2) for details on plant transpiration, canopy conductance,
    	         and canopy evaporation.
    	\item[-] See section 3.1 and equations 6 -- 7 for how bare soil evaporation is calculated.
    	\item[-] Equations 2 -- 4 for soil heat flux calculations. Setting two values to 0 can remove
    	         the impact of snow. The warm season coefficient is discussed more in section 3.2.
    	\item[-] Figure 16. Dewpoint still needs work.
    	\item[-] Interestingly, transpiration and evaporation seems to be calculated from plant and
    	         soil conditions separately, then summed together. They are derived from a form of the
    	         Penman equation. Thus, they are derived from the energy balance.
    	\item[-] Upgrading and improving snow pack and frozen soil physics is important for representing
    	         wintertime conditions.
    	\item[-] Noted some under prediction in plant transpiration by the Noah LSM.
    	\begin{itemize}
    		\item[-] In offline (non-coupled) studies.
    	\end{itemize}
        \item[-] The Noah LSM is also being used in the WRF.
    \end{itemize}

    \underline{Chen et al. 1996 -- Transpiration Methods}
    \begin{itemize}
    	\item[-] Equation 3 for a diffusive form of Richard's equation (prognostic equation for soil
    	         moisture).
    	\begin{itemize}
    		\item[-] Section 3.1.1 and paragraph with this equation for explanation of variables.
    	\end{itemize}
        \item[-] Equations 9 -- 11, and 13 -- 14, and Table 2 for canopy transpiration calculations and
                 corresponding paragraph (pages 4 -- 5).
        \item[-] Figures 2 -- 15 for comparisons with field data and to see some of the model output
                 and performance.
        \item[-] Tested against field experiment in northeastern Kansas.
        \item[-] OSU model generally performed well, though runoff had somewhat more mixed results.
    \end{itemize}

    \underline{Wang et al. 2018 -- Sensitivity of Ensemble Models to Bias}
    \begin{itemize}
    	\item[-] Quantification of predictability of the atmosphere is the absolute error of the forecast
    	         (e.g., root mean square error (RMSE))
    	\item[-] Uncertainty is measured through ensemble spread (standard deviation, with respect to the
    	         ensemble mean).
    	\item[-] Range of possible solutions associated with a NWP is the ensemble distribution.
    	\item[-] Ensemble spread is used to simulate RMSE of ensemble mean forecast, in both magnitude
    	         and spatial structure.
    	\item[-] In a perfect ensemble prediction, the spread and RMSE are equal (in magnitude an perfect
    	         correlation in space).
    	\item[-] The relationship between the forecast error and spread is called the ``spread-skill-
    	         relationship.''
    	\item[-] A common problem is under dispersion (ensemble spread is significantly smaller than
    	         mean forecast error).
    	\item[-] Models are biased due to model inherent deficiencies and errors in initial conditions
    	         (Toth et al. 2003).
    	\item[-] Removing the bias can enhance ensemble forecast performance (paragraph 2, page 2, several
    	         citations to support this).
    	\item[-] Du 2007 -- An ensemble system is designed to only deal with random errors, but not
    	         systematic errors.
    	\item[-] Equations 1 -- 2 and associated paragraphs (pages 2 -- 4) for bias correction methods.
    	\item[-] Figures 1 \& 3 show how well removing the bias helps.
    	\item[-] Figure 2 and pages 6 \& 7 on how bias was identified and how much.
    	\item[-] While the magnitudes were much improved from debiasing, the spatial correlation was not
    	         much improved.
    	\item[-] Removing the bias should be done prior to verifying the ensemble forecast.
    	\item[-] Page 10 for how the CRPS score works.
    	\item[-] Surface variables are generally less uniform in space than upper air (did this really
    	         need to be told?).
    	\item[-] The bias removal improves the probabilistic forecast.
    	\item[-] Bias should be removed before verification to prevent inaccurate interpretations and
    	         conclusion.
    	\item[-] Note that the bias removed is the bulk bias, not the flow bias.
    	\begin{itemize}
    		\item[-] Bulk bias effects the ensemble mean position in the distribution. Flow bias effects
    		         the distribution itself.
    	\end{itemize}
        \item[-] It would likely be useful to recreate Figure 2 and the analysis in section $3b$ for ET,
                 PET, and SESR in the GEFS to help determine any model bias and assess the skill (most
                 likely have to use NARR as the truth value).
    \end{itemize}

	
    \underline{Zhou et al. 2017 -- GEFS Configuration}
    \begin{itemize}
    	\item[-] GEFS was implemented in 1993.
    	\item[-] Note for the GEFS the spatial resolution changes after the 8 day forecast.
    	\item[-] Also note the v10 GEFS has a 6 hour time resolution, and 3 hour time resolution for v11
    	         (unsure which is currently operational; v11 changes to 6 hour resolution after the 8 day
    	         forecast).
    	\item[-] Under dispersive is still a problem.
    	\item[-] Continuous verification method includes:
    	\begin{itemize}
    		\item[-] Continuous ranked probability score (CRPS)
    		\begin{itemize}
    			\item Measures the area difference between cumulative distributions of forecasts and
    			      observations
    		\end{itemize}
    	    \item[-] RMSE/spread
    	    \item[-] MERR (mean error/absolute error)
    	\end{itemize}
        \item[-] Categorical verification methods include:
        \begin{itemize}
        	\item[-] Brier score/Brier skill score (BS/BSS)
        	\item[-] Reliability
        	\item[-] Bias
        	\item[-] Equitable threat score (ETS)
        	\item[-] True skill score (TSS)
        \end{itemize}
        \item[-] Note these methods are for QPF forecasts, but see page 8 for details.
    \end{itemize}


	\underline{Hao et al. 2018 -- Drought Prediction Overview}
	\begin{itemize}
		\item[-] 
	\end{itemize}
\end{document}