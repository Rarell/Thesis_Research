% This document will be a rough draft for a thesis outline, including a title discussed
% in a meeting on Sepember 3, and some key words to go with it.

\documentclass[12pt, letterpaper]{article}
\usepackage{syntonly}
% Next three packages are on the off chance that graphs or mathematics are needed in here.
\usepackage{amsmath}
\usepackage{amssymb}
\usepackage{graphicx}

%\syntaxonly % Use this when adding new lines to check for compile errors.
% 			   (increases run speed and prevents document being produced.)
\title{Identification and Predictability of Flash Drought using the Global Forecast%
       System} % Title from group discussion on September 3. Note Ryyan was the big
               % contributor towards this title (it is mostly from one of her suggestions)
\date{} % Omit the date
               
\begin{document}
  \maketitle % Add the title

  \begin{enumerate}
    \item Introduction 
    \begin{enumerate}
      \item Background information
      \begin{itemize}
        \item[-] May include discussion of need for concrete definition for flash drought.
        \item[-] Definition is addressed via SESR and introduction of criteria here.
        \item[-] Other pertinent information. (Determining where GFS stands on drought
                 forecasts may be useful here.)
      \end{itemize}
      
      \item Address reason/purpose for the study.
      \begin{itemize}
        \item[-] Impacts of flash drought and need push "hindcasting" of flash drought
                 to nowcasting or forecasting in order to adapt to them and mitigate impacts.
        \item[-] Research questions here.
        \begin{itemize}
          \item[$\rightarrow$] Move identification of flash drought from hindsight to foresight.
          \item[$\rightarrow$] Determine how well the GFS identifies and forecasts flash drought.
                             (identification and forecast may be considered separately.)
          \item[$\rightarrow$] Create a product that can be used in the GFS to forecast flash
                             drought (more of a long term goal or future product).
        \end{itemize}
      \end{itemize}
    \end{enumerate}
      
    \item Literature review
    \begin{enumerate}
      \item Review of literature on flash droughts
      \item Review of literature on model progress and forecast attempts on droughts
      \item Review of literature on GFS and where it currently stands
    \end{enumerate}

    \item Data
    \begin{enumerate}
      \item GFS data or GEFS data (ensemble will require bias correction)
      \begin{itemize}
        \item[-] Likely legacy and FV3
        \item[-] Region is likely over CONUS. Extra code work if over NA or is that a
                 processing issue?
        \item[-] TBD which models run(s) (e.g., 12Z run, 00Z run, etc.; At which hour does it
                 reinitialize, it if does?).
        \item[-] Some uncertainty here if working with the model itself, or the output data.
      \end{itemize}
    \end{enumerate}

    \item Methods
      \begin{enumerate}
        \item Identify flash drought with the GFS (i.e., realtime analysis).
        \begin{itemize}
          \item[-] Some questions to address here:
          \begin{itemize}
            \item[$\rightarrow$] Determining whether to use pentads vs. window analysis or
                                 another type of analysis.
            \item[$\rightarrow$] Note a synthetic dataset of past data will need to be
                                 created for identification (similar to the Kyle Griffin
                                 page, but for a longer time period)
          \end{itemize}
        \end{itemize}
        \item Forecast/identify risk area for flash drought (1 week out, 2 weeks, etc.).
        \begin{itemize}
          \item[-] Should become trivial after identification (move time step foreward 
                   while using same methodology for identification with synthetic dataset
                   created above)
          \item[-] Need to address inconsistency in temporal and spatial resolution after
                   the 10 day forecast (deterministic) or 8 day forecast (ensemble)
          \item[-] Could be identification or percentage of risk forecast.
          \item[-] ``Predictability'' needs to be identified and determined here.
        \end{itemize}
      \end{enumerate}
    
    \item Results
      \begin{enumerate}
        \item TBD following actual analysis.
        \item Prediction: First forecast will need much improvement.
      \end{enumerate}
    
    \item Discussion
      \begin{enumerate}
        \item Does TBD need to be said anymore?
      \end{enumerate}

    \item Conclusion
    \begin{enumerate}
      \item See above three sections for this answer.
    \end{enumerate}
    
  \end{enumerate}
  
  Some key words: Flash drought, GFS, drought prediction, ET, PET, SESR (maybe), LSM,
  Land - Atmosphere

\end{document}